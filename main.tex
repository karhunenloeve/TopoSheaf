\documentclass[leqno]{article}
\usepackage[ngerman]{babel}
\usepackage{a4wide}
\usepackage{tikz}
\usepackage{tikz-cd}
\usetikzlibrary{babel}
\usepackage{xcolor}
\usepackage{amsmath,amsfonts,amssymb,amsthm,epsfig,epstopdf,titling,url,array}
\usepackage{url}

\theoremstyle{plain}
\newtheorem{thm}{Satz}[section]
\newtheorem{lem}[thm]{Lemma}
\newtheorem{prop}[thm]{Proposition}
\newtheorem*{cor}{Korollar}

\theoremstyle{definition}
\newtheorem{defn}{Definition}[section]
\newtheorem{conj}{Behauptung}[section]
\newtheorem{exmp}{Beispiel}[section]

\theoremstyle{remark}
\newtheorem*{rem}{Anmerkung}
\newtheorem*{note}{Note}

\usepackage{enumitem}
\setlist[itemize]{noitemsep, nolistsep}

\title{Natürliche Transformationen, Äquivalenzen von Kategorien, darstellbare Funktoren und das Lemma von Yoneda}
\author{Luciano Melodia}
\date{25.10.2022}

\begin{document}

\maketitle

\begin{abstract}
In diesem Vortrag behandeln wir die Definition von natürlichen Transformationen. Wir veranschaulichen natürliche Transformationen durch möglichst viele Beispiele, insbesondere solche, die für die Topologie von besonderem Interesse sind. Dann befassen wir uns mit der Frage, wann Kategorien als äquivalent angesehen werden können. Schließlich klären wir den Begriff der Darstellbarkeit für kovariante und kontravariante Funktoren.

Mit Hilfe dieser Begriffe wird es uns gelingen, das Lemma von Yoneda zu beweisen, das oft als das wichtigste mathematische Ergebnis innerhalb der Kategorientheorie angesehen wird. Es beschreibt die Menge der natürlichen Transformationen zwischen einem $\textbf{Hom}$-Funktor und einem beliebigen anderen Funktor.
\end{abstract}

\section{Natürliche Transformationen}
Im Folgenden seien $\mathcal{A},\mathcal{B}$ Kategorien, und $F,G: \mathcal{A} \rightarrow \mathcal{B}$ Funktoren. Um natürliche Transformationen definieren zu können, brauchen wir die folgenden Daten und Eigenschaften:
\begin{itemize}
	\item Einen Morphismus $\tau_{X}: F(X) \rightarrow G(X)$ für jedes Objekt $X \in \mathcal{A}$.
	\item Die Eigenschaft falls $f: X \rightarrow Y$ ein Morphismus in $\mathcal{A}$ ist, so gilt
	\begin{equation}
		G(f) \circ \tau_{X} = \tau_{Y} \circ F(f).
	\end{equation}
\end{itemize}
Dies führt uns zu folgender Definition.

\begin{defn}[Natürliche Transformationen]
Eine \textbf{natürliche Transformation} $\tau: F \Longrightarrow G$ von einem Funktor $F$ zu einem zweiten Funktor $G$ ordnet jedem Objekt $X \in \mathcal{A}$ einen Homomorphismus $\tau_{X}: F(X) \rightarrow G(X)$ in $\mathcal{B}$ zu, auch \textbf{Komponente von $\tau$ bei $X$} genannt, so dass für jeden Morphismus $f: X \rightarrow Y$ in $\mathcal{A}$ das folgende Diagramm kommutiert:
\end{defn}
\begin{equation}
	\begin{tikzcd}
	F(X) \arrow[rr, "F(f)" description] \arrow[d, "\tau_{X}" description] &  & F(Y) \arrow[d, "\tau_{Y}" description] \\
	G(X) \arrow[rr, "G(f)" description]                                   &  & G(Y)                                  
	\end{tikzcd}.
\end{equation}
Als Formel lässt sich das Diagramm ausdrücken durch folgende Gleichung: $\tau_{Y} \circ F(f) = G(f) \circ \tau_{X}$. Für Funktoren $F,G$ wird mit $\text{Mor}_{\text{Fun}}(F,G)$ die \textbf{Menge der natürlichen Transformationen} von $F$ nach $G$ bezeichnet.

\begin{rem}
Die natürliche Transformation $\tau$ ist die Gesamtheit aller Morphismen $\tau_{X}$. In der Literatur findet sich diese Tatsache in der Notation $\tau = (\tau_{X})_{X \in \mathcal{A}}$ wieder, wobei jedes $\tau_{X}$ eine Komponente von $\tau$ ist. Man kann eine gewisse Analogie zu dem Aufbau von Folgen als Gesamtheit ihrer Terme $s = (s_n)_{n \in \mathbb{N}}$ sehen.
\end{rem}

In anderen Worten ist eine natürliche Transformation eine Sammlung von Abbildungen von einem Diagramm in ein anderes. Das Besondere an diesen Abbildungen ist, dass sie kommutieren. Nehmen wir eine natürliche Transformation zwischen zwei Funktoren $\tau: F \Longrightarrow G$, dann können wir durch folgendes Diagramm eine bessere Intuition für natürliche Transformationen gewinnen:
\begin{equation}
	\begin{tikzcd}
	                                                   & {\color{red}F} \arrow[r, Rightarrow, "\tau"]                 & {\color{blue}G}, &                            &                 \\
	{\color{red}F(X)} \arrow[dd, red] \arrow[rd, red] \arrow[rrr, "\tau_{X}"] &                                         &   & {\color{blue}G(X)} \arrow[rd, blue] \arrow[dd, blue] &                 \\
	                                                   & {\color{red}F(Z)} \arrow[ld, red] \arrow[rrr, "\tau_{Z}", gray] &   &                            & {\color{blue}G(Z)} \arrow[ld, blue] \\
	{\color{red}F(Y)} \arrow[rrr, "\tau_{Y}"]                       &                                         &   & {\color{blue}G(Y)}                       &                
	\end{tikzcd}.
\end{equation}
Um das obige Diagramm in Gänze verstehen zu können, betrachten wir im Anschluss ein paar Sonderfälle und Beispiele für natürliche Transformationen.

\begin{exmp}
\begin{enumerate}
	\item Die Funktoren $F,G: \mathcal{A} \rightarrow \mathcal{B}$ seien beide konstante Funktoren, also wird jedes Objekt $X$ in $\mathcal{A}$ auf ein einziges Objekt $F(X)$ in $\mathcal{B}$ abgebildet und jeder Morphismus auf $\text{Id}_{F(X)}$ in $\mathcal{B}$. Analog soll $G$ ein Objekt und einen Morphismus auf ein festes $F(Y)$ und $\text{Id}_{F(Y)}$ in $\mathcal{B}$ abbilden. Dann ist eine natürliche Transformation von $F$ nach $G$ gegeben durch $\tau: F(X) \Longrightarrow F(Y)$.
	\item Sei $F: \mathcal{A} \rightarrow \mathcal{B}$ konstant für ein Objekt $X$ in $\mathcal{A}$ und $G: \mathcal{A} \rightarrow \mathcal{B}$ ein beliebiger Funktor. Dann besteht die natürliche Transformation $\tau: F \Longrightarrow G$ aus Abbildungen $\tau_{X}: F(X) \rightarrow G(X)$, eine für jedes $X \in \mathcal{A}$, so dass $\tau_Y = G(f) \circ \tau_X$, falls $f:X \rightarrow Y$ ein Morphismus in $\mathcal{A}$ ist. Veranschaulichen kann man das durch folgendes kommutierende Diagram, 
	\begin{equation}
		\begin{tikzcd}
		               & F(X) \arrow[lddd, "\tau_X" description, bend right] \arrow[dd, "\tau_V" description] \arrow[rdd, "\tau_W" description, bend left] \arrow[ddd, "\tau_Y" description, bend left] &                 \\
		               &                                                                                                                                                                             &                 \\
		               & G(V) \arrow[r] \arrow[ld]                                                                                                                                                   & G(W) \arrow[ld] \\
		G(X) \arrow[r, "G(f)"] & G(Y)                                                                                                                                                                        &                
		\end{tikzcd},
	\end{equation}
	wobei $X,Y,V,W \in \mathcal{A}$ sind und die Knoten und Kanten des unteren Vierecks das durch $G$ gegebene Diagramm meint. Die Gleichung $\tau_Y = G(f) \circ \tau_X$, die erfüllt werden muss, bringt jeweils die Seiten im Tetraeder zum kommutieren. In diesem Fall heißt $\tau$ \textbf{Kegel über $G$}.
	\item Sei andererseits $G$ konstant bei $X \in \mathcal{A}$, dann besteht eine natürliche Transformation aus einer Familie von Abbildungen $\tau_X: F(X) \rightarrow G(X)$, so dass $\tau_Y \circ F(f) = \tau_X$ gilt, genau dann wenn $f: X \rightarrow Y$ ein Morphismus in $\mathcal{A}$ ist. Veranschaulicht durch ein kommutierendes Diagramm,
	\begin{equation}
		\begin{tikzcd}
		                                                         & F(V) \arrow[ld] \arrow[r] \arrow[ddd, "\tau_V" description, bend right, shift left = 0.5] & F(W) \arrow[ld] \arrow[lddd, "\tau_W"', bend left] \\
		F(X) \arrow[r, "F(f)"] \arrow[rdd, "\tau_X", bend right] & F(Y) \arrow[dd, "\tau_Y"', bend left]                                                &                                                    \\
		                                                         &                                                                                      &                                                    \\
		                                                         & G(X)                                                                                 &                                                   
		\end{tikzcd}
	\end{equation}
	müssen die Seiten des Diagramms kommutieren. Dieses Diagramm heißt \textbf{Kegel unter $F$}, oder manchmal auch \textbf{Kokegel von $F$} (eng. co-cone).
	\item Seien $F,G: \mathcal{A} \rightarrow \mathcal{B}$ Funktoren und $f: X \rightarrow Y$ ein Morphismus in $\mathcal{A}$. Falls jede Komponente $\tau_X: F(X) \rightarrow G(X)$ von $\tau$ ein Isomorphismus ist, dann ist die Bedingung $\tau_Y \circ F(f) = G(f) \circ \tau_X$ äquivalent zu $F(f) = \tau_y^{-1} \circ G(f) \circ \tau_X$, da $\tau_Y$ invertierbar ist. In einer Kategorie ist der Isomorphismusbegriff so definiert, dass für einen Morphismus $f:X \rightarrow Y$ und $g:Y \rightarrow X$ gilt $g \circ f = \text{Id}_X$ und $f \circ g = \text{Id}_Y$, genau dann wenn $X$ und $Y$ isomorph sind. In einem kommutierenden Diagramm lässt sich das wie folgt darstellen:
	\begin{equation}
				\tau_Y \circ F(f) = G(f) \circ \tau_X, \hspace{1.5cm} F(f) = \tau_Y^{-1} \circ G(f) \circ \tau_X,
	\end{equation}
	\vspace{-0,5cm}
	\begin{equation}
		\begin{tikzcd}
		F(X) \arrow[r, "F(f)"] \arrow[d, "\tau_X"] & F(Y) \arrow[d, "\tau_Y"] &  & F(X) \arrow[r, "F(f)"] & F(Y)                           \\
		G(X) \arrow[r, "G(f)"]                     & {G(Y),}                  &  & G(X) \arrow[u, "\tau_X^{-1}"] \arrow[r, "G(f)"]       & G(Y). \arrow[u, "\tau_Y^{-1}"]
		\end{tikzcd}
	\end{equation}
	Wenn also jedes $\tau_X$ ein Isomorphismus ist, ist die Natürlichkeitsbedingung vergleichbar mit der Konjugation. Sie erinnert auch an eine Homotopie von $G$ nach $F$. Beide Sichtweisen legen nahe, dass, wenn jedes $\tau_X$ ein Isomorphismus ist, $F$ und $G$ bis auf Umbenennung ihrer Elemente wirklich derselbe Funktor sind. Wenn dies der Fall ist, wird die natürliche Transformation $\tau$ als \textbf{natürlicher Isomorphismus} bezeichnet, und $F$ und $G$ werden als \textbf{natürlich isomorph} bezeichnet.
\end{enumerate}
\end{exmp}

Ohne Zweifel ist man bereits über den Begriff des (Co)limes in der Kategorientheorie gestoßen. In den obigen Beispielen der Kegel und Kokegel werden bereits einige (Co)limites beschrieben. Beispiele wären die leere Menge, die Einpunktmenge, der Schnitt, die Vereinigung, das Produkt von Mengen, der Kern einer Gruppe, der Quotient von topologischen Räumen, die direkte Summe zweier Vektorräume, das freie Produkt von Gruppen oder der Pullback über ein Faserbündel. Jedes ist ein Spezialfall, welches eine Art universellen Kegel über einen bestimmten Funktor bildet.

\subsection{Die Kategorie $B\textbf{G}$}
Wir betrachten eine Gruppe $G$.

\begin{prop}
Dann ist $B\textbf{G}$ mit einem einzigen Objekt $G$ und einem Morphismus $g: G \rightarrow G$ für jedes Element $g \in G$ zusammen mit der Verknüpfung $\circ: G \times G \rightarrow G$ eine Kategorie. Die Verknüpfung der Gruppe ist auch die Verknüpfung innerhalb von $B\textbf{G}$.
\end{prop}

\begin{proof} Da eine Gruppe abgeschlossen ist unter der Komposition, gilt für zwei Morphismen $g,h: G \rightarrow G$, dass auch $g \circ h: G \rightarrow G$ ein Morphismus ist, nämlich das Gruppenelement $gh \in G$. Die Identität $e \in G$ fungiert als Identitätsmorphismus für $G$ und die Assoziativität gilt aufgrund dessen, da die Verknüpfung der Gruppe bereits assoziativ ist.
\end{proof}

Nun können wir einen Funktor $F: B\textbf{G} \rightarrow \textbf{Set}$ in die Kategorie der Mengen definieren, der jedes Objekt $G \in B\textbf{G}$ auf genau eine Menge $M \in \textbf{Set}$ abbildet, und jedes Gruppenelement $g \in G$ auf eine Funktion $g \cdot - : M \rightarrow M, m \mapsto g \cdot m$ abbildet.

\begin{prop}
Die Menge der Funktoren $Fun(B\textbf{G}, \textbf{Set})$ ist äquivalent zu $G-\textbf{Set}$.
\end{prop}

\begin{proof}
Seien $C,D: B\textbf{G} \rightarrow \textbf{Set}$ Funktoren, sodass $C(G) = M$ und $D(G) = N$ für $M,N \in \textbf{Set}$ und sei $g: G \rightarrow G$ ein Element der Gruppe. Dann ist $\tau: C \Longrightarrow D$ eine natürliche Transformation, bestehend aus genau einer Funktion $\tau: M \rightarrow N$, sodass $\tau(g(m)) = g(\tau(m))$, für jedes $m \in M$.
\end{proof}

Natürliche Transformationen sind also \textbf{$G$-äquivariante Abbildungen}. Diese Gleichung lässt sich mit dem nachfolgenden kommutierenden Diagramm noch einmal veranschaulichen.
	\begin{equation}
		\begin{tikzcd}
		M \arrow[d, "g\cdot-"] \arrow[r, "\tau"] & N \arrow[d, "g\cdot-"] & m \arrow[d, maps to] \arrow[r, maps to] & \tau(m) \arrow[d, maps to] \\
		M \arrow[r, "\tau"]                      & {N,}                   & gm \arrow[r, maps to]                   & \tau(g(m)) = g(\tau(m)).  
		\end{tikzcd}
	\end{equation}

\section{Äquivalenzen von Kategorien}
Äquivalente Kategorien sind identisch, mit Ausnahme der Tatsache, dass sie eine unterschiedliche Anzahl von isomorphen Kopien eines Objekts enthalten können. Wir werden diese Aussage in diesem Abschnitt präzisieren, indem wir definieren, wann zwei Kategorien als äquivalent gelten. Zu diesem Zweck seien $\mathcal{A}, \mathcal{B}$ Kategorien. 

Man könnte die Äquivalenz von Kategorien wie folgt definieren:

\begin{defn}[Äquivalente Kategorien]
Die Kategorien $\mathcal{A}$ und $\mathcal{B}$ heißen \textbf{äquivalent}, wenn es Funktoren $F: \mathcal{A} \rightarrow \mathcal{B}$ und $G: \mathcal{B} \rightarrow \mathcal{A}$ gibt, so dass der Funktor $F \circ G$ äquivalent ist zu $\text{Id}_{\mathcal{B}}$ und der Funktor $G \circ F$ äquivalent ist zu $\text{Id}_\mathcal{A}$.
\end{defn}

Diese Definition ist jedoch nicht besonders nützlich. Es ist nicht sinnvoll, so etwas wie eine Inverse von Funktoren zu definieren, da sich die Suche nach einer solchen Inversen in der Regel als äußerst schwierig erweist. Die folgende Definition schafft daher Abhilfe:

\begin{defn}[Treue und volltreue Funktoren]
Ein Funktor $F: \mathcal{A} \rightarrow \mathcal{B}$ wird als
\begin{enumerate}
	\item \textbf{treu} bezeichnet, wenn für alle Objekte $X,Y \in \mathcal{A}$ die durch $F$ induzierte Abbildung von der Menge der Morphismen in $\mathcal{A}$ in die der Kategorie $\mathcal{B}$, $\text{Mor}_\mathcal{A}(X,Y) \rightarrow \text{Mor}_\mathcal{B}(F(X),F(Y))$, injektiv ist.
	\item \textbf{volltreu} bezeichnet, wenn für alle Objekte $X$ und $Y$ in $\mathcal{A}$ die durch $F$ induzierte Abbildung $\text{Mor}_\mathcal{A}(X,Y) \rightarrow \text{Mor}_\mathcal{B}(F(X),F(Y))$ eine Bijektion ist.
\end{enumerate}
\end{defn}

\begin{defn}[Äquivalenz von Kategorien]
Ein Funktor $F: \mathcal{A} \rightarrow \mathcal{B}$ heißt \textbf{Äquivalenz von Kategorien}, wenn er volltreu und surjektiv auf Isomorphieklassen ist, also wenn es für jedes $Y \in \mathcal{B}$ ein $X \in \mathcal{A}$ mit $F(X) \cong Y$ gibt. Die Kategorien $\mathcal{A}$ und $\mathcal{B}$ heißen \textbf{äquivalent}, wenn eine Äquivalenz $F: \mathcal{A} \rightarrow \mathcal{B}$ existiert.
\end{defn}

\begin{exmp}
Sei $\mathcal{A}$ die Kategorie mit nur einem Objekt $X$ und nur einem Morphismus $\text{Id}_X: X \rightarrow X$. Sei $\mathcal{B}$ die Kategorie mit genau zwei Objekten $\widetilde{X}, \widetilde{Y}$ und den Morphismen $\text{Id}_{\widetilde{X}}:\widetilde{X} \rightarrow \widetilde{X}, \text{Id}_{\widetilde{Y}}: \widetilde{Y} \rightarrow \widetilde{Y}, a: \widetilde{X} \rightarrow \widetilde{Y}$ und $b: \widetilde{Y} \rightarrow \widetilde{X}$ mit Verkettungen $a \circ b = \text{Id}_{\widetilde{Y}}$ und $b \circ a = \text{Id}_{\widetilde{X}}$. 
\begin{conj}
Die Kategorien $\mathcal{A}$ und $\mathcal{B}$ sind äquivalent.
\end{conj}
\begin{proof}
Sei $F: \mathcal{A} \rightarrow \mathcal{B}$ ein Funktor mit $X \mapsto \widetilde{X}$, so dass $F(\text{Id}_X) = \text{Id}_{\widetilde{X}}$.
\begin{enumerate}
	\item Dann ist die durch $F$ induzierte Abbildung $\text{Mor}_\mathcal{A}(X,X) \rightarrow \text{Mor}_{\mathcal{B}}(F(X),F(X))$ injektiv, denn $\text{Mor}_{\mathcal{B}}(\widetilde{X},\widetilde{X})$ ist eine Ein-Element-Menge. Der Funktor $F$ ist also treu.
	\item Es gilt auch, dass die durch $F$ induzierte Abbildung $\text{Mor}_\mathcal{A}(X,X) \rightarrow \text{Mor}_{\mathcal{B}}(\widetilde{X},\widetilde{X})$ surjektiv ist, denn
	\begin{equation}
		\text{Mor}_{\mathcal{B}}(\widetilde{X},\widetilde{X}) := \{\text{Id}_{\widetilde{X}}\},
	\end{equation}
	ist eine Ein-Element-Menge. Also ist $F$ volltreu.
	\item Es bleibt zu zeigen, dass für jedes $\widetilde{X} \in \mathcal{B}$ ein $X \in \mathcal{A}$ existiert, so dass $F(X) \cong \widetilde{X}$. Also sei $F(X) = \widetilde{X}$, dann gilt $F(X) \cong \widetilde{X}$ trivial. Da aber $a \circ b = \text{Id}_{\widetilde{Y}}$ und $b \circ a = \text{Id}_{\widetilde{X}}$, gilt auch $\widetilde{X} \cong \widetilde{Y}$. Aufgrund der Tatsache, dass Isomorphie eine Äquivalenzrelation auf Klassen definiert, gilt auch $F(X) \cong \widetilde{Y}$.
\end{enumerate}
Dies zeigt, dass $\mathcal{A}$ und $\mathcal{B}$ äquivalent sind.
\end{proof}
\end{exmp}

\section{Darstellbare Funktoren}
Sei $\mathcal{A}$ eine Kategorie und $F: \mathcal{A} \rightarrow \textbf{Set}$ ein vergesslicher Funktor.

\begin{defn}[Darstellbare Funktoren]
Der Funktor $F$ heißt \textbf{darstellbar}, wenn es ein Objekt $X$ in $\mathcal{A}$ gibt und einen Isomorphismus $F \cong \text{Mor}_{\mathcal{A}}(X,\cdot)$ von Funktoren.
\end{defn}

Anders ausgedrückt soll der Funktor $F: \mathcal{A} \mapsto \text{Mor}_\mathcal{A}(X,\cdot)$ abbilden, also $F(M) = \text{Mor}_\mathcal{A}(X,M)$ in $\textbf{Set}$. Man beachte, dass der Funktor $F$ kovariant ist. Man betrachte dafür nur den Morphismus $f: M \rightarrow N$, dann ist der induzierte Morphismus $F(f): F(M) \rightarrow F(N)$ gegeben durch $\text{Mor}_{\mathcal{A}}(X,M) \rightarrow \text{Mor}_{\mathcal{A}}(Y,M)$. Die Fragen, die wir mit der Darstellbarkeit klären wollen, lassen sich also explizit wie folgt formulieren:

\begin{itemize}
	\item Sei eine Kategorie $\mathcal{A}$ gegeben. Was ist dann $F$ und was ist $M$?
	\item Sei ein Funktor $F: \mathcal{A} \rightarrow \text{Sets}$ gegeben, ist dann $F \cong \text{Mor}_{\mathcal{A}}(X,\cdot)$ für ein $X \in \mathcal{A}$?
\end{itemize}

Sei $f: X \rightarrow Y$ ein Morphismus. Für jedes $M$ erhalten wir auf diese Weise eine Abbildung
\begin{align}
(\cdot \circ f)_M: \text{Mor}_{\mathcal{A}}(Y,M) &\rightarrow \text{Mor}_{\mathcal{A}}(X,M), \\
g &\mapsto g \circ f.
\end{align}

Das liefert uns eine natürliche Transformation $(\cdot \circ f): \text{Mor}_{\mathcal{A}}(Y,\cdot) \rightarrow \text{Mor}_{\mathcal{A}}(X,\cdot)$.

\begin{exmp}
	\begin{enumerate}
		\item Sei $U: \textbf{Grp} \rightarrow \textbf{Set}$ ein Funktor von der Kategorie der Gruppen in die Kategorie der Mengen. Damit wird eine Gruppe auf die zu Grunde liegende Menge abgebildet und der Gruppenhomomorphismus auf die zu Grunde liegende Funktion.
		Wir betrachten ein Element $g \in G$, dann gibt es einen eindeutigen Homomorphismus $\phi_g: \mathbb{Z} \rightarrow G$, so dass $1 \mapsto g$. Andererseits wird $1$ immer auf ein Element von $G$ abgebildet, also handelt es sich hierbei um eine Bijektion zwischen der Menge an Elementen in $G$ und der Menge an Homomorphismen von $\mathbb{Z}$ nach $G$. Diese Bijektion $\tau_G: U(G) \rightarrow \text{Mor}_{\textbf{Grp}}(\mathbb{Z},G)$ sind also die Komponenten des natürlichen Isomorphismus. Es bleibt zu zeigen, dass für jeden Gruppenhomomorphismus $\varphi: G \rightarrow H$ für Gruppen $G,H$ die Gleichung
		\begin{equation}
		\tau_H \circ U(\varphi) = \text{Mor}_{\textbf{Grp}}(\mathbb{Z},\varphi) \circ \tau_G
		\end{equation}
		erfüllt ist. Wir zeichnen zunächst ein kommutierendes Diagramm:
		\begin{equation}
			\begin{tikzcd}
			U(G) \arrow[rr, "U(\varphi)"] \arrow[d, "\tau_G"]                                                     &  & U(H) \arrow[d, "\tau_H"]                 \\
			{\text{Mor}_{\textbf{Grp}}(\mathbb{Z},G)} \arrow[rr, "{\text{Mor}_{\textbf{Grp}}(\mathbb{Z},\varphi)}"] &  & {\text{Mor}_{\textbf{Grp}}(\mathbb{Z},H)}.
			\end{tikzcd}
		\end{equation}
		Da $U(\varphi)$ lediglich die dem Gruppenhomomorphismus zu Grunde liegende Funktion ist, sendet die linke Seite der Gleichung ein Element $g \in G$ durch den eindeutigen Homomorphismus $\phi_{\varphi(g)}: \mathbb{Z} \rightarrow H$, so dass $\phi_{\varphi(g)}(1)=\varphi(g)$. Auf der rechten Seite bilden wir zunächst $g$ auf den Homomorphismus $\phi_g:\mathbb{Z} \rightarrow G$ ab und verknüpfen diesen dann mit $\varphi$. Dies erfüllt ebenfalls die Gleichung $\varphi(\phi_g(1)) = \varphi(g)$ und damit $\varphi \circ \phi_g = \phi_{\varphi(g)}$. Also wird $U$ durch $\mathbb{Z}$ repräsentiert.
	\end{enumerate}
\end{exmp}



\section{Yoneda Lemma}
Möchte man mathematische Objekte besser verstehen, so postuliert das Yoneda-Lemma, ist es notwendig die Beziehung dieser Objekte zu allen anderen Objekten mit denselben mathematischen Eigenschaften zu verstehen. Ein Objekt wird also vollständig durch seine Beziehungen zu anderen Objekten bestimmt, d.h. dadurch, wie das Objekt aus der Sicht jedes Objekts in der Kategorie \emph{aussieht}. Im Folgenden präzisieren wir diese Aussage durch das Yoneda-Lemma:

\begin{lem}[Das Yoneda-Lemma]
Seien $X$ und $Y$ Objekte einer lokal kleinen Kategorie $\mathcal{A}$ und $\tau: \text{Mor}_{\mathcal{A}}(Y,\cdot) \rightarrow \text{Mor}_{\mathcal{A}}(X,\cdot)$ ein Isomorphismus von vergesslichen Funktoren. Dann gibt es genau einen Isomorphismus $f: X \rightarrow Y$ mit $\tau = (\cdot \circ f)$. Insbesondere ist das Objekt $X$ durch den Funktor $\text{Mor}_{\mathcal{A}}(X,\cdot)$ bestimmt bis auf eindeutige Isomorphie.
\end{lem}

\begin{proof}
Sei $\tau: \text{Mor}_{\mathcal{A}}(Y,\cdot) \rightarrow \text{Mor}_{\mathcal{A}}(X,\cdot)$ ein Isomorphismus von vergesslichen Funktoren. Wir erinnern uns an dieser Stelle, dass $\text{Mor}_{\mathcal{A}}(Y,\cdot)$ eine Menge von vergesslichen Funktoren $F: \mathcal{A} \rightarrow \textbf{Set}$ ist. Sei $f: X \rightarrow Y$ das Bild von $\text{Id}_Y$ und $M$ ein weiteres Objekt in der Kategorie $\mathcal{A}$ und $\phi \in \text{Mor}_\mathcal{A}(Y,M)$. Dann gibt es ein kommutatives Diagramm
\begin{equation}
	\begin{tikzcd}
	{\text{Mor}_{\mathcal{A}}(Y,Y)} \arrow[d, "\phi \circ"] \arrow[r, "\tau_Y"] & {\text{Mor}_{\mathcal{A}}(X,Y)} \arrow[d, "\phi \circ"] \\
	{\text{Mor}_{\mathcal{A}}(Y,M)} \arrow[r, "\tau_M"]                         & {\text{Mor}_{\mathcal{A}}(X,M)}.                        
	\end{tikzcd}
\end{equation}
Wenn wir oben links mit der Identität starten, erhalten wir über den linken unteren Weg $\tau_M(\phi)$, über den oberen rechten Weg aber $\phi \circ f$. Also ist $\tau_M(\phi) = \phi \circ f$ für alle Daten $(M,\phi)$ und somit erhalten wir die natürliche Transformation als Komposition: $\tau = (\cdot \circ f)$. Die Abbildung $f:X \rightarrow Y$ ist aber auch die einzige Abbildung, die diese Gleichung erfüllt. Das erkennt man, indem man $M = Y$ und $\phi = \text{Id}$ wählt: $\tau_Y(\text{Id}) = \text{Id} \circ f = f$, wie im folgenden Diagramm zu sehen ist:
\begin{equation}
	\begin{tikzcd}
	{\text{Mor}_{\mathcal{A}}(Y,Y)} \arrow[d, "\phi \circ"] \arrow[r, "\text{Id}"] & {\text{Mor}_{\mathcal{A}}(X,Y)} \arrow[d, "\phi \circ"] \\
	{\text{Mor}_{\mathcal{A}}(Y,Y)} \arrow[r, "\text{Id}"]                         & {\text{Mor}_{\mathcal{A}}(X,Y)}.                        
	\end{tikzcd}
\end{equation}
Analog zu obiger Konstruktion liefert $\tau^{-1}: \text{Mor}_{\mathcal{A}}(X,\cdot) \rightarrow \text{Mor}_{\mathcal{A}}(Y,\cdot)$ eine Abbildung $g := \tau^{-1}_X (\text{Id}_X): Y \rightarrow X$.
Zuletzt zeigen wir noch das $f$ ein Isomorphismus ist.


Wir betrachten nun folgendes Diagramm, das nach Definition von natürlichen Transformationen kommutativ ist:
\begin{equation}
	\begin{tikzcd}
	{\text{Mor}_{\mathcal{A}}(Y,Y)} \arrow[d, "g \circ"] \arrow[r, "\tau_Y"] & {\text{Mor}_{\mathcal{A}}(X,Y)} \arrow[d, "g \circ"] \\
	{\text{Mor}_{\mathcal{A}}(Y,X)} \arrow[r, "\tau_X"]                      & {\text{Mor}_{\mathcal{A}}(X,X)}.                     
	\end{tikzcd}
\end{equation}
Wenn wir oben links mit $\text{Id}_Y$ beginnen, so erhalten wir, wenn wir nach unten und dann nach rechts abbilden, die Identität auf $X$, nach Definition von $g$. Bilden wir die Identität aber nach rechts und dann nach unten ab, so erhalten wir zuerst $f$ und dann $g \circ f$. Aus der Kommutativität folgt also $\text{Id}_X = g \circ f$. Analog erkennen wir, dass $\text{Id}_Y = f \circ g$. Die Abbildung $f:X \rightarrow Y$ ist also ein Isomorphismus.
\end{proof}

\begin{thebibliography}{9}
\bibitem{A} Peter Fiebig (2020) \emph{Einführung in die Darstellungstheorie. Vorlesungsskript}, Friedrich-Alexander Universität Erlangen-Nürnberg, Erlangen.
\bibitem{B} Tai-Danae Bradley, Tyler Bryson, and John Terilla (2020) \emph{Topology. A Categorical Approach}, MIT Press, Cambridge.
\bibitem{B} Tai-Danae Bradley (2017) \emph{What is a Functor? Definition and Examples, Part I + II}, URL: \url{https://www.math3ma.com/blog/what-is-a-functor-part-1}, eingesehen am 17.20.2022.
\bibitem{B} Tai-Danae Bradley (2017) \emph{What is a Natural Transformation? Definition and Examples, Part I + II}, URL: \url{https://www.math3ma.com/blog/what-is-a-natural-transformation-1}, eingesehen am 17.20.2022.
\end{thebibliography}
\end{document}
